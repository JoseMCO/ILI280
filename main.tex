\documentclass[12pt,letterpaper,hidelinks]{article}
\usepackage[utf8]{inputenc}
\usepackage[spanish]{babel}
\usepackage{amsmath}
\usepackage{amssymb}
\usepackage{graphicx}
\usepackage{enumerate}
\usepackage{listings}
\usepackage[left=1.5cm,right=1.5cm,top=2cm,bottom=1.5cm]{geometry}
\renewcommand\spanishtablename{Tabla}
\lstset{ 
  basicstyle=\scriptsize,
  breaklines=true,
}

\author{José Castro (201273514-9)}
\title{Lab1 estaca}

%Hipervinculos y Glosario
\usepackage{hyperref}


%Color
\usepackage{color}
\definecolor{nc1}{RGB}{16, 89, 82}
\definecolor{nc2}{RGB}{23, 127, 117}
\definecolor{nc3}{RGB}{33, 162, 168}
\usepackage{sectsty}
%\sectionfont{\color{nc1}}
%\subsectionfont{\color{nc2}}
%\subsubsectionfont{\color{nc3}}

%Cabeceras
\usepackage{fancyhdr}
\pagestyle{fancy}
\fancyhead[L]{ILI-280}
\fancyhead[C]{Ingeniería Civil Informática - Plan 7300}
\fancyhead[R]{UTFSM}
\setlength{\headheight}{14.5pt}

\usepackage{tikz}
\usetikzlibrary{babel,arrows,shapes,automata,positioning}
\usepackage{fontspec}
\begin{document}


\begin{titlepage}
 
\begin{center}
\includegraphics[width=.2\textwidth]{utfsm}\hspace{.5\textwidth}
\includegraphics[width=.2\textwidth]{inf}
\\[1cm]
{\Large \textsc{Universidad Técnica Federico Santa María} }\\[.5cm]
{\Large {ILI-280 Estadística Computacional} }\\[.5cm]
{\Large {Segundo Semestre 2016} }\\[1.5cm]
{\Huge {Laboratorio 1}}\\[10cm]

\begin{minipage}[l]{0.4\textwidth}
	\begin{flushleft}
		\textbf{\textsf{Profesor:}}\\
		{\large Ricardo Ñanculef}\\ 
		\textbf{\textsf{Ayudante:}}\\
		{\large Ignacio Loayza}\\ 
		\end{flushleft}
\end{minipage}
\begin{minipage}[l]{0.4\textwidth}
	\begin{flushright}
		\textbf{\textsf{Nombre:}}\\
		\large José Miguel Castro\\
		\textbf{\textsf{Rol:}}\\
		\large 201273514-9
	\end{flushright}
\end{minipage} 
\end{center}
 
\end{titlepage}

\section{Medidas de Tendencia y dispersión.}

Es común que las instituciones educacionales, tales como universidades, realicen encuestas u otras formas de recopilación de datos que les permitan inferir sobre la calidad y efectividad de su modelo educativo. A continuación se trabajará con datos recabados para una muestra aleatoria de estudiantes de la UTFSM.

\begin{enumerate}[\hspace{.5cm} a)]

\item Realice una tabla resumiendo a grandes rasgos el dataset.
\begin{table}[ht]
\centering
\begin{tabular}{|rlllll|}
  \hline
 & Total & Sexo & Horas.Estudio.Semanal & VTR &    Tiempo.Libre \\ 
  \hline
1 & 870 & F:501   & 2-5 hr :409   & 0:724   & Demasiado: 89   \\ 
  2  & & M:369   & 5-10 hr:136   & 1:113   & Mucho    :237   \\ 
  3  & &  & $<$2 hr  :266   & 2: 33   & Nada     : 61   \\ 
  4  & &  & $>$10 hr : 59   &  & Normal   :330   \\ 
  5  & &  &  &  & Poco     :153   \\ 
  6  & &  &  &  &  \\ 
  7  & &  &  &  &  \\ 
   \hline
\end{tabular}
\end{table}

\begin{table}[ht]
\centering
\begin{tabular}{|rllll|}
  \hline
 &      Carrete &        Salud & Inasistencias & Nota.Final \\ 
  \hline
1 & Demasiado:137   & Buena     :148   & 0      :179   & $<$55 :325   \\ 
  2 & Mucho    :188   & Muy Buena :306   & 2      :159   & $>$=55:545   \\ 
  3 & Nada     : 62   & Muy Mala  :121   & 4      :114   &  \\ 
  4 & Normal   :272   & Normal    :190   & 10     : 97   &  \\ 
  5 & Poco     :211   & Suficiente:105   & 6      : 92   &  \\ 
  6 &  &  & 3      : 69   &  \\ 
  7 &  &  & (Other):160   &  \\ 
   \hline
\end{tabular}
\end{table}

\item Encuentre el máximo y mı́nimo para la variable VTR.
\begin{table}[ht]
\centering
\begin{tabular}{|rr|}
  \hline
 & VTR \\ 
  \hline
Min. & 0.00 \\ 
  Max. & 2.00 \\ 
   \hline
\end{tabular}
\end{table}

\item Encuentre la cantidad de personas que reprobaron los ramos con nota <55 y la frecuencia relativa asociada.

\begin{table}[!ht]
\centering
\begin{tabular}{|rr|}
  \hline
 & Reprobados \\ 
  \hline
Cantidad & 325 \\ 
  Rel.Freq & 0.37 \\ 
   \hline
\end{tabular}
\end{table}

\clearpage

\item Construya un nuevo dataset que incluya solo a aquellos alumnos que reprobaron los ramos con nota <55

\begin{enumerate}[i)]
\item ¿Cuántas personas dicen haber mantenido un estado de salud ’Bueno’ o ’Muy Bueno’ y cuantas un estado de salud ’Malo’ o ’Muy Malo’?

\begin{table}[!ht]
\centering
\begin{tabular}{|rr|}
  \hline
 & Salud \\ 
  \hline
Buena & 176 \\ 
  Mala &  77 \\ 
   \hline
\end{tabular}
\end{table}

\item ¿Cuántas personas dicen haber estudiado 2-5 Hr semanalmente?

\begin{table}[!ht]
\centering
\begin{tabular}{|rr|}
  \hline
 & Horas \\ 
  \hline
2-5 & 159 \\ 
   \hline
\end{tabular}
\end{table}

\item ¿Cuántas personas tuvieron una cantidad de inasistencias menor a 5, un estado de salud ’Bueno’ o ’Muy Bueno’ y dedicaron al estudio de su asignatura a lo menos 5-10 Hr?

\begin{table}[!ht]
\centering
\begin{tabular}{|rr|}
  \hline
 & Buenos \\ 
  \hline
Cantidad &  14 \\ 
   \hline
\end{tabular}
\end{table}

\item Analice y concluya lo que usted cree que pueden indicar sus cálculos.
\\
Una de las conclusiones que pueden ser sacadas de este último cálculo es que la cantidad de alumnos reprobados que fueron "responsables" fue mínima, ya que esta corresponde sólo a un 1.6\% del total, lo que implica que, asistiendo a clases y estudiando más de 5 horas a la semana hay mayores probabilidades de aprobar el ramo.

\begin{table}[!ht]
\centering
\begin{tabular}{|rr|}
  \hline
 VTR & Inasistencias \\
  \hline
1 & 3.57 \\
2 & 5.33 \\
3 & 5.30 \\ 
   \hline
\end{tabular}
\end{table}

Además, como se muestra en la tabla anterior, se puede ver que los alumnos que toman el ramo por primera vez tienden a asistir más a clases, teniendo más inasistencias los alumnos que inscriben por segunda o tercera vez el ramo.

\end{enumerate}

\item Construya una tabla de resumen de la variable VTR.
\begin{table}[!ht]
\centering
\begin{tabular}{|rr|}
  \hline
 & VTR \\ 
  \hline
Median & 0.00 \\ 
  Mean & 0.21 \\ 
  3rd Qu. & 0.00 \\ 
  Max. & 2.00 \\ 
  Std.Dev. & 0.49 \\ 
  Var & 0.24 \\ 
   \hline
\end{tabular}
\end{table}

\pagebreak

\item Obtenga sub-muestras de tamaño n = 200, 500 y 700 para la variable Inasistencias, encuentre la media de cada sub-muestra.

\begin{table}[!ht]
\centering
\begin{tabular}{|rr|}
  \hline
 & Mean.SubMuestras \\ 
  \hline
200 & 3.43 \\ 
  500 & 3.76 \\ 
  700 & 3.86 \\ 
   \hline
\end{tabular}
\end{table}

\end{enumerate}

\pagebreak

\section{Representación Gráfica de Datos.}

Una parte importante de la estadística consiste en buscar formas simples y concisas de presentar la información recogida de las mediciones con el fin de que otros puedan entender el trabajo realizado

\begin{enumerate}[\hspace{.5cm} a)]

\item Construya un gráfico de barras para la distribución de frecuencias de cada uno de los estados de salud de la variable Salud.

\begin{figure}[h]
\centering
\includegraphics[width=.45\textwidth]{Gsalud}
\label{img:Gsalud}
\end{figure}

\item Obtenga dos sub-muestras de datos, una que contenga sólo mujeres y la otra sólo hombres, para cada una de estas sub-muestras construya un boxplot para la variable Inasistencias.

\begin{figure}[h]
\centering
\includegraphics[width=.45\textwidth]{GinasistenciasMF}
\label{img:Ginasistencias}
\end{figure}

Como se puede ver claramente en el gráfico, la mediana es muy similar para ambos sexos y se encuentra cercana al valor 3.86, lo que quiere decir, que el 50\% de los alumnos tienen a lo menos 3 inasistencias. Además se puede ver que para ambos se presenta un sesgo,  el que representa una preferencia a una menor cantidad de inasistencias. Junto a lo anterior, se puede notar una mayor variabilidad para los hombres, ya que la distancia entre el primer y el tercer cuartil es mayor que para las mujeres.

\item Realice un histograma de la variable Inasistencias utilizando la regla de Sturges para calcular el numero de clases del histograma.

\begin{figure}[h]
\centering
\includegraphics[width=.45\textwidth]{Ginasistencias}
\label{img:Ginasistencias}
\end{figure}

\begin{table}[!ht]
\centering
\begin{tabular}{|rrrr|}
\hline
Intervalos & Marca Clase & Frecuencia Relativa & Frecuencia Acomulada \\ \hline
{[}0-1{]} & 0.5 & 0.24942529 & 217 \\
(1-2{]} & 1.5 & 0.18275862 & 376 \\
(2-3{]} & 2.5 & 0.07931034 & 445 \\
(3-4{]} & 3.5 & 0.13103448 & 559 \\
(4-5{]} & 4.5 & 0.06091954 & 612 \\
(5-6{]} & 5.5 & 0.10574713 & 704 \\
(6-7{]} & 6.5 & 0.01149425 & 714 \\
(7-8{]} & 7.5 & 0.05287356 & 760 \\
(8-9{]} & 8.5 & 0.01494253 & 773 \\
(9-10{]} & 9.5 & 0.11149425 & 870 \\
\hline
\end{tabular}
\end{table}

\pagebreak

\item Para finalizar, investigue y realice al menos dos conclusiones generales sobre los datos que involucren cualquiera de las medidas y gráficos que obtuvo.

\begin{table}[!ht]
\centering
\begin{tabular}{|rr|}
  \hline
Carrete & Inasistencias \\ 
  \hline
Demasiado & 3.93 \\ 
  Mucho & 4.47 \\ 
  Normal & 3.64 \\ 
  Poco & 3.60 \\ 
  Nada & 3.76 \\ 
   \hline
\end{tabular}
\end{table}

Como se puede ver en la tabla anterior, contrariamente a lo que se pueda pensar, la cantidad de "carreteo" de los alumnos no tiene relación con la asistencia en clases.

\pagebreak

\section{Anexo}

\subsection{Código}

\lstinputlisting{Lab1.R}

\end{enumerate}
\end{document}